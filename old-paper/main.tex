
\documentclass[conference]{IEEEtran}

\IEEEoverridecommandlockouts 
\usepackage{macros}

\usepackage{amsmath}
\usepackage{blindtext, graphicx}
\usepackage{algorithm}

\usepackage{bbm}
\usepackage{soul}

\usepackage{algpseudocode}
\usepackage{pifont}
\usepackage{tikz}
\usepackage{amssymb}

\usepackage{upgreek}

\usepackage{mathtools}



\ifCLASSINFOpdf
 
\else
  
\fi

\hyphenation{Smart grids}

\begin{document}



\title{Agent based decision making system for stochastic supply \& demand management in inter connected microgrid networks}




\author{
\IEEEauthorblockN{D. Sai Koti Reddy,\\
Krishnasuri Narayanam}
\IEEEauthorblockA{IBM Research - India\\
Email: saikotireddy@in.ibm.com}
\and
\IEEEauthorblockN{D. Raghuram Bharadwaj,\\
Shalabh Bhatnagar}
\IEEEauthorblockA{Department of Computer Science and Automation\\
Indian Institute of Science, 
Bangalore, India}
}

\maketitle


\begin{abstract}


One of the key challanges of Smart Grid management is ensuring to meet the demand while having control on the supply cost. \st{This concept of Smart Grid has become popular in the recent times. The main objective in this Smart Grid is to intelligently make use of power.}  It involves \st{taking} performing optimal actions both at the production and consumption sites of \st{electricity} electric grid. In this work, we consider the problem of managing multiple microgrids attached to a central smart grid. Each of these microgrids are equipped with batteries to store renewable power. \st{In this work, we consider multiple microgrids equipped with batteries to store renewable power}. At every instant, each of them receive a demand to meet. Depending on the supply (i.e., currently available battery energy, power drawn either from the central grid or from the peer microgirds), each of the microgrids take a decision on from where to draw the energy to meet the demand. \st{Depending on the current battery and renewable power information, they take a decision on number of power units to be bought or sold.} When a microgrids buys energy either from the central grid or from the peer microgrids, it can use that energy either to meet the current demand or to store in the battery storage for future use. \st{If any of the other neighboring microgrids sell the power, they can consume it. Otherwise, they can get it from the main grid. If power is bought, it is first used to meet its demand and rest of it will be stored in the battery.} Hence, there is a control decision problem at each microgrid on the number of power units to be bought or sold at every time instant. We note that both the forecasted demand and predicted renewable supply impact this decision by each microgrid. \st{We note that the future forecast demand and renewable units also impact this decision.} Further, we consider some amount of the forecasted demand to be adjustable in terms of the time when it can be met by the microgrid. Such an adjustable demand is attributed due to the activities of daily living pertraining to Smart Homes connected to the microgrid networks. Hence, we formulate this problem in the framework of Markov Decision Process and apply Reinforcement Learning algorithms to solve this problem. Through simulations, we show that the policy we obtain performs an significant improvement over traditional techniques.


\end{abstract}

%\section{Introduction}


Research on smartgrids can be classified into two areas -  Demand-side management and Supply-side management. Demand side management (DSM) (\cite{logenthiran2011multi, wang2010demand,dsm1,dsm2,dsm3,dsm4}) deals with techniques developed to efficiently use the power by bringing the customers into the play. The main idea is to reduce the consumption of power during peak time and shifting it during the other times. This is done by dynamically changing the price of power and sharing this information with the customers. 

%In \cite{reddy2011learned}, Reinforcement Learning (RL) is used in smart grids for pricing mechanism so as to improve the profits of broker agents who procure energy from power generation sources and sell it to consumers.  
Supply-side management deals with developing techniques to efficiently make use of renewable and non-renewable energy at the supply side. We now discuss the papers that are close to our work. In \cite{PHarsha}, authors consider the problem of optimal energy storage management problem under dynamic cost setup. They consider a renewable generator that is equipped with a limited storage battery and capable of meeting the some local demands. It also has connection from the main grid. The decision to be taken at every instant is the number of power units to be stored in the battery. They allow this to be negative as power can be drawn from the battery. They formulate this problem as a Markov Decision Process under long run average cost. The objective is to minimize the long run average cost of power bought from the main grid.

The assumption made in this paper is that demand at all times can be met. This is facilitated by allowing the main grid as many units of power as asked by the renewable generator. In our work, we extend this MDP model to put additional constraint on the maximum number of power units that a main grid can provide. Another notable contribution is that we extend this setup to the multiple microgrids. Here, we allow the microgrids to share the power among themselves.  

In \cite{reddy2011learned}, they consider the problem of maximizing the profits among the broker agents. These agents buy the power from the producers and sell it to the customers. They apply Reinforcement Learning (RL) algorithms to solve this problem to show that learning policies perform better than the traditional non-learning strategies.

In \cite{goodmdp}, authors propose an MDP for solving the problem of minimizing the demand and supply deficit  
and apply dynamic optimization methods. But when the model information (the renewable energy generation in this case) is not known, we cannot apply these techniques.

\section*{Organization of the Paper}	
The rest of the paper is organized as follows. The next section describes the important problems associated with the microgrids and solution techniques to solve them. Section III presents the results of experiments of our algorithms. Section IV provides the concluding remarks and Section V discusses the future research directions.
\section{Introduction}

A microgrid is a networked group of distributed energy sources with the goal of generating, converting and storing energy. While the main power stations on central Smart Grid are highly connected, micro-grids with local power generation, storage and conversion/transmission capabilities, act locally or share power with a few neighboring micro-grid nodes \cite{farhangi2010path}.

In order to take full advantage of the modularity and flexibility of micro-grid technologies, smart control mechanisms are required to manage and coordinate these distributed energy systems so as to minimize the costs of energy production, conversion and storage, without jeopardizing the central smart grid stability to which these micro-grids are connected to. Augmenting micro-grid with smart controls however involves addressing many problems. In this paper, we address two  problems. (i) Supply-side management (SSM) problem: energy sharing among the microgrids under stochastic supply \& demand along with the optimal battery scheduling of each microgrid (ii) Demand-side management (DSM) problem: efficiently scheduling the time adjustable demand from smart appliances in the smart home environment along with the normal demand. Our goal here is to reduce the energy demand and supply deficit in the long-run. We address these learning and scheduling problems by modeling them as a Markov decision process (MDP) \cite{puterman2014markov}.

\subsection{Supply-side management problem}
Cooperative energy exchange among microgrids is a popular technique in SSM for efficient energy distribution.  Local energy sharing/exchange between microgrids has the following advantages:
(a) it can significantly reduce power wastage that would otherwise result over long-distance transmission lines, and (b) it helps satisfy demand and reduce resiliance on the main grid. Figure~\ref{gridmodel} shows a cooperative energy exchange model with multiple microgrids that can cater to their individual local loads. Each microgrid controls its local sub-network through its controller (labelled $\mbox{C}_1$, $\mbox{C}_2$ etc.) that mainly has access to its local state information.


\begin{figure}[thpb]
      \centering
      \includegraphics[scale=0.4]{powergrid2.pdf}
      \caption{Cooperative Energy Exchange Model}
      \label{gridmodel}
\end{figure}

In classical power grids, system level optimization is done based on a centralized objective function, where as microgrid network has heterogeneous nature right from the manner in which electricity is generated such as from wind turbines, solar farms and diesel generators to energy storage devices such as batteries and capacitors. Because of this heterogeneity and the fact that energy can be shared between microgrids depending on requirements, one needs to consider asynchronous distributed techniques 
%such as multi-agent reinforcement learning or game theory
to control and optimize a smart grid system connected to microgrid distribution networks.

\subsection{Demand-side management problem}
 Load shifting is a popular technique used in demand-side management (DSM) \cite{DTU2010}. It involves moving the consumption of load to different times within an hour, or within in a day, or even within a week. It doesn't lead to reduction in net quantity of energy consumed, but simply involves changing the time when the energy is consumed. Advantage due to load shifting for the customer is reduction in the energy consumption cost, and the advantage for the smart grid is in managing the peak load consumption. Hence load shifting is beneficial for both the consumers and the smart grid.

With the increased use of the smart appliances and smart home environments, the concept of load shifting is becoming increasingly handy for the smart grid as the demand from smart appliances is time adjustable in general. One or more of these smart appliances collectively achieve some activity in the smart home environment, called as an ADL (activity of daily living). It's possible to monitor and identify the ADLs in the smart home environments \cite{I2014, GPG2016}. When an ADL is active, the smart appliances associated with that ADL are switched on to perform the activity defined by the ADL thus adding load on the smart grid. With the help of the smart home technology, it's possible to find the amount of load each ADL puts on the grid, and also the allowed time window during which the ADL would perform the activity (e.g., washing machine running for an hour to clean the cloths anytime between 3PM to 6PM). If the time window for the ADL lets the smart grid have more than one possible way of scheduling the load, it's considered as flexible ADL. On the other hand, if the time window for the ADL lets the smart grid have exactly one possible way of scheduling the load, it's considered as non-flexible ADL (e.g., washing machine running for an hour to clean the cloths anytime between 3PM to 4PM, is not flexible since there is only one option of switching on the washing machine at 3PM). Thus the demand from the flexible ADLs need not be met at a fixed time period, instead could be met at any time period within a flexible time window. With the help of the advanced metering infrastructure (AMI) \cite{RAFK2014} that provides a two-way communication between the utility and customers, it's possible to take the decision of when to schedule the ADL demand at the smart grid and convey the same to the customer's smart meter.    

There is other regular demand that needs to be met at fixed time periods, apart from the zero or more ADL related demand associated with any customer. This regular demand along with the zero or more non-flexible ADL demand of a smart home is considered to be non-ADL demand for the rest of the paper. Similarly, the demand due to zero or more flexible ADLs of the smart home is considered to be ADL demand.
 
There is prior art around scheduling the ADL-demand using the load shifting technique for handling the peak load scenarios \cite{CL2014}. However, they precisely know the supply profile while doing such a scheduling of the ADL-demand. In this paper, we propose scheduling of ADL-demand using the load shifting technique with uncertainty in the supply profile generated (e.g., renewable energy sources like solar or wind being the primary sources of power generation).

\textbf{Related work :} \cite{saad2012game} provides a survey on game theoretic approaches for microgrids where both cooperative energy sharing models as well as non-cooperative game models for distributed control of microgrids are examined when the system model is known. Since  models for energy dynamics are very unreliable \cite{zamora2010controls}, one has to use model-free algorithms to address these problems.  Because of their model-free nature, reinforcement learning \cite{sutton1998reinforcement} approaches that are primarily data-driven control techniques are playing a significant role in these problems.

In \cite{zifadistributed}, distributed reinforcement learning algorithm for coordinated energy sharing and voltage restoration in a islanded DC microgrid is proposed. In \cite{leo2014reinforcement}, reinforcement learning algorithm for optimal battery scheduling under the dynamic load environment and sloar power is proposed with the goal of  reducing  energy consumption from the main grid. In this paper, we  consider the coordinated energy sharing among the grid connected microgrids with optimal battery scheduling problem when stochastic supply and adjustable stochastic demand is available.

\textbf{Our contributions :}
\begin{inparaenum}[\bfseries (i)]
We summarize our contributions as follows :\\
\item To the best of out knowledge, we are the first one to integrate both the Demand-side and Supply-side management problems  in a single Markov decision process framework. We used reinforcement learning algorithms which do not require knowledge of the underlying model to address these problems. Our algorithms are easy to implement and also scalable.\\
\item The Optimal scheduling of ADL demand at microgrid level, where both the demand and power generation is stochastic is first time introduced through this work. \\    
\end{inparaenum}
\section*{Organization of the Paper}	
The rest of the paper is organized as follows. The next section describes the important problems associated with the microgrids and solution techniques to solve them. Section III presents the results of experiments of our algorithms. Section IV provides the concluding remarks and Section V discusses the future research directions.


\IEEEpeerreviewmaketitle

\section{Problem formulation and mdp model}



\subsection{MDP model}

The decision taken by the microgrid $i$ at the time $t$ is denoted as $u_{t}^{i}$ and $v_{t}^{i}$.
Let the set of ADL jobs at microgrid $i$ at time $t$ be $J_{t}^{i}$. And  $J_{t}^{i}= \{\gamma_{1}^{i},\ldots,\gamma_{n}^{i}\}$, where jth ADL job $\gamma_{j}^{i} = (a_{j}^{i}, y_{j}^{i})$ consists of the number of units of power required to finish the job (denoted by  $a_{j}^{i}$) and number of time slots remaing to shedule the  job (denoted by  $y_{j}$).
The state space be $s_{t}^{i} = (t,nd_{t}^{i},p_{t}, J_{t}^{i})$ where $nd_{t}^{i} = r_{t}^{i} + b_{t}^{i} - d_{t}^{i}$. If this is negative, it means there is a deficit in demand and positive implies there is excess of power.
Let $P_{t}^{i} = \{\Gamma_{1}^{i},\ldots,\Gamma_{N}^{i}\}$ be the power set of $J_{t}^{i}$, which consists of all possible combinations of the ADL jobs that can be sheduled at time instant $t$ at microgrid $i$. 
Let  $A_{t}^{i} = \{A(\Gamma_{1}^{i}),\ldots,A(\Gamma_{N}^{i})\} $, where $A(\Gamma_{j}^{i}) = \sum_{\gamma_{k}^{i} \in \Gamma_{j}^{i} } a_{k}^{i}$.


Then the action is bounded as follows. For ease of understanding, we drop the subscripts.  

\begin{align}
-min(M, B - nd + &\max_{1\leq j \leq N} A(\Gamma_{j}^{i}) ) \leq u \nonumber\\ &\leq max(0, nd - \min_{1\leq j \leq N} A(\Gamma_{j}^{i}))
\end{align}

The intuition behind the bounds is as follows. A microgrid can sell atmost the excess power. That is, the power remaining after meeting the demand. While buying, it can buy to meet the demand and also to fill its battery.

Now after getting $u_{t}^{i}$, we construct the feasible set $F_{t}^{i}$, which is a subset of $P_{t}^{i}$. Each element $\Gamma_{j}^{i}$ in the $F_{t}^{i}$ has to satisfy the following condition $A(\Gamma_{j}^{i}) \leq u_{t}^{i} $.
Agent has to pick the action $v_{t}^{i} = \Gamma_{j}^{i} \in F_{t}^{i}$. Now ADL jobs in $\Gamma_{j}^{i}$ will get sheduled. Let $J_{t+1}^{i}$ be the new ADL jobs at time instatant $t+1$. $J_{t+1}^{i} = J_{t+1}^{i} \cup \widetilde J_{t}^{i}$, where $\overline J_{t}^{i} = J_{t}^{i} - v_{t}^{i}$ and $\widetilde J_{t}^{i} =  \{(a_{1}^{i}, y_{1}^{i} - 1),\ldots,(a_{n}^{i}, y_{n}^{i} - 1)\}$, where $ (a_{j}^{i}, y_{j}^{i}) \in \overline J_{t}^{i}$.

The battery information is updated as follows:

\begin{align}
b_{t+1}^{i} = max(0,nd_{t}^{i} - u_{t}^{i})
\end{align}

We formulate the single stage cost function as follows :
\begin{align}
g^{i}(s,u) = p_{t}*u_{t}^{i} + c*(min(0,nd_{t}^{i} - u_{t}^{i})) + c* \sum_{k =1}^{n} I_{y_{k}^{i} = 0} a_{k}^{i}.
\end{align}

The first term represents the cost of buying or selling the power and the second term represents the demand and supply deficit. For every unit of demand that is not met, the microgrid incurs a cost of $c$. 

\subsection{Average cost setting}
\section{Algorithm}\label{sec:algo}

%We first note that the renewable generation is uncertain in nature. 
%That is, we do not know in the current time period, the renewable generation in the future time periods.

 In this paper, we do not assume any model of the system (i.e., probability transition model of the demand, supply and reward structure) due to uncertainity of renewable energy generation. We employ RL agorithms which do not assume any model to provide optimal solution.

We employ the Q-Learning algorithm, a  popular RL method for solving the average cost problem in section \ref{subsec:avg}.
%To solve the above average cost problem, we apply a popular RL algorithm, Q-Learning.
 Our objective is to obtain an optimal policy $\pi^{*}$.
We apply the Relative Value Iteration (RVI) based Q-Learning algorithm described in \cite{avgcost}. In this algorithm, we update the Q-values in each iteration according to the following rule:
\begin{align}
Q^{n+1}(&s,u) = Q^{n}(s,u) + \alpha(n)(g(s,u,s^{'}) + \nonumber\\ &  max_{u} Q^{n}(s^{'},u) - max_{u} Q^{n}(s_{0},u) - Q^{n}(s,u)),
\end{align}
where $\alpha$ is the learning rate, $g(s,u,s^{'})$ is the reward obtained by taking an action $u$ in the state $s$ and transitioning to the state $s^{'}$ and $s_{0}$ is any prescribed state.
Also, $Q^n(s,u)$ represents the $n$th estimate of the Q-value  obtained in state $s$ by taking action $u$. In \cite{avgcost}, it is shown that under appropriate learning rate, the algorithm converges to the  optimal policy. 
Each microgrid runs a version of this algorithm independently until convergence. The optimal policy of microgrid $i$ is obtained as follows:
\begin{align}
\pi_{i}^{*}(s) = max_{u}Q^{i}(s,u),
\end{align}
that is, the optimal action in state $s$ is obtained by taking the maximum over all actions of the Q-values in state $s$.  


\section{Simulation Experiments}\label{sec:experiments}
We implemented our models on a network with three microgrids. Two of them  operate on solar renewable generation and the other on wind energy. To simulate the renewable generation, we use RAPsim software \cite{rapsim}. RAPsim is an open source simulator for analyzing the power flow in microgrids. It has a provision for simulating the renewable generation, which is the main feature that we use in our experiments. We construct our microgrid model as shown in Fig 2. We can see that there are three microgrids, two of them operating on the solar energy and the other on the wind energy. The solar microgrid in the right has more capacity than that of the one in the left. These microgrids also have electrical connections from the main grid. Each microgrid provides power to the respective houses on their power line. 


\begin{figure}[thbp] \label{exp}
	\centering
	\includegraphics[scale = 0.6]{experimental_setup.jpg}
		\caption{Experimental Setup}
\end{figure}


\subsection{Implementation}

We implement the model we described in the section 2. We call this model as $ADL-sharing$ model. For comparison purposes, we implement following models. 

\begin{itemize}
	\item \textbf{Greedy-ADL model}: In this model, the microgrids will exhibit the greedy behavior. They share the power only if there is a excess power after filling up the battery. The action in each instant is bounded by   
	
	
	\begin{align}
	-min(M_t^i, B_t^i &- nd_t^i + \max_{1\leq j \leq 2^n} A(j) ) \leq u_t^i \nonumber\\ &\leq max(0, nd_t^i - B_t^i - \min_{1\leq j \leq 2^n} A(j)),
	\end{align}
	
	
	
	That is, if the net demand is negative, decision is taken on amount to power to buy to meet the demand and fill the battery. On the other hand, if the net demand is positive, it is first used to fill the battery and only if there is any excess power left, it will be sold to the other microgrids.
	
	\item \textbf{Non-ADL model}:  This model is similar to the $ADL-sharing$ model, but without the concept of ADL jobs. In this model, the ADL demand is included in the main demand. Unlike the $ADL-sharing$ model, there is no flexibility of intelligently scheduling the ADL jobs. 
	
\end{itemize}

\subsection{Setup}

% \begin{figure}
%still need the trimming of figure. include when we put experimental figures
% \includegraphics[scale = 0.4]{experimental_setup.jpg}
% \end{figure}

We simulate the above setup for the month of September 2017 in the RAPsim and collect the wind and solar renewable power generated each day every hour. Using this data, we fit poisson distribution and obtain the poisson mean. The parameters for our experiments are described below. The number of decision time periods is taken to be 4 (i.e., t = 4). We consider 3 demand values for all the microgrids - 2, 4 and 6 units. The probability transition matrix for all the 3 microgrids are given below :


\[
P_{1}=
\begin{bmatrix}
0.2 & 0.6 & 0.2 \\
0.1 & 0.2 & 0.7 \\
0.8 & 0.1 & 0.1
\end{bmatrix}
\]

\[
P_{2}=
\begin{bmatrix}
0.2 & 0.2 & 0.6 \\
0.8 & 0.1 & 0.1 \\
0.2 & 0.7 & 0.1
\end{bmatrix}
\]

\[
P_{3}=
\begin{bmatrix}
0.5 & 0.5 & 0 \\
0 & 0.5 & 0.5 \\
1 & 0 & 0
\end{bmatrix}
\]




The price values is considered to be 5, 10 and 15. The Probability transition matrix for the price vector is given below:



\[
Q=
\begin{bmatrix}
0.2 & 0.4 & 0.4 \\
0.1 & 0.5 & 0.4 \\
0.5 & 0.4 & 0.1
\end{bmatrix}
\]


Maximum size of battery and renewable power generated is taken to be 8 units. The maximum power that a microgrid can obtain from the main grid is set to 10 units.

We consider 3 ADLs in our experiment. We assume that all these 3 ADL's are known to microgrids in the first time period. First ADL requires 1 unit of power that needs to be satisfied within the second time period. Second ADL requires 1 unit of power within the third time period. Third ADL requires 2 units of power within the fourth time period.

With this setup, we compare our proposed models. The algorithms are trained for $10^6$ iterations. For comparison purposes, we plot value of threshold $c$ on X-axis and Average reward on Y-axis. Average reward is computed as follows. We run the trained models for 1000 runs and average the reward obtained by each microgrid. For the second figure, we average the profit obtained by each microgrid for every 100 iterations.

\begin{figure}[thbp] \label{r1}
	\centering
	\includegraphics[scale = 0.2]{first_plot.jpg}
	\caption{Comparison of Models on 3 microgrids}
\end{figure}

\begin{figure}[thbp] \label{r1}
	\centering
	\includegraphics[scale = 0.2]{second_plot.jpg}
	\caption{Comparison of models across iterations for $price = 10$  }
\end{figure}


\subsection{Observations}
\begin{itemize}

\item Consider the case of $c = 0$. Note that this doesn't mean that the agents do not satisfy the demand of the customers. It just means that the agents don't incur penalty for not satisfying the excess demand. The agents need not buy the power to satisfy the excess demand. In the $ADL-sharing$ model, we observe that all the agent fills the battery when the price is low and sells the power when the price is high. Hence there will be not much sharing among the agents. Also, the profit obtained is very high compared to the other models. This is because in $Greedy-ADL$ model, the power bought will be first stored in the battery and the only te excess will be sold. In $Non-ADL$ model, the demand values are higher compared to other models and hence there will very less excess power to sell and make profits.

\item When $c >0$, we observe the sharing among the agents. This is because, at any time instant each microgrid has different configurations of current and future demand and renewable resources. An agent operating on solar renewable generation in the time period 2 share the excess power with the other agents, as it is generates more supply as the day progresses. A the same time, agent operating on wind renewable generation buys the power to store in its battery, if it expects more demand than it generates in future time period. This results in the sharing of power among the agents.   
	
\item With ADL jobs being included along with the non-ADL demand, our RL algorithm schedules few of the ADLs at the beginning of the allowed execution time window, few others are scheduled at the end of the allowed execution time window of the ADL, while some other ADLs get scheduled at the mid of the allowed execution time window. This ensures that the RL learning agents exploit the fact that the ADL demand is flexible to meet in a given range of time window. %On the other-hand, it is not desired that the learning agent schedules all the ADLs either at the beginning or at the end of the allowed time window of execution.

\item When surplus energy is available at a microgrid at any time instant, we observe that the microgrids do not sell all of this to the other microgrids if there is more demand than supply in the future. For example, if the renewable energy source for a microgrid is solar energy, then if there is surplus energy (i.e., excess energy available after meeting the demand at some moment) at the microgrid during the midday, the microgrid sell that surplus energy to the other microgrids (because it is expected to generate more supply as the day progresses); on the other-hand, if there is surplus energy at the microgrid during the end of the day, the microgrid do not sell all of that surplus energy to the other microgrids (because there may not be much supply possible for the rest of the day).

\item In Figure 4, we observe that as the number of iterations increases, the performance of the learning algorithm improves. Moreover, the model 2 and the model 3 converges faster than that of our model 1, because of larger state and action space in model 1. 


%\item How are we ensuring this? If there is 5 units of surplus at time t. If the demand at time (t+1) is 5 units, it's possible to meet that demand by storing 5 units at time t. Other possibility is, sell the 5 units in time t, and buy 5 units in time $t+1$ from some other microgrid. However the first option is most desired. How are we ensuring this in our experiments? One possible way to implement this is by ensuring the buying cost to be more than the selling cost for one unit of energy.   
\end{itemize}

\subsection{Discussion}

In Figure 3, we compare our $ADL-sharing$ model with the $Greedy- ADL$ and $Non-ADL$. As discussed earlier, in the $Greedy-ADL$ model the sharing of power is done only when there is excess after filling the battery. We thus see that the first model outperforms the second model. Even though there will be less buying of power in $Greedy-ADL$, there will be no selling of power as well. Therefore the overall profit obtained will not be higher than that, when the intelligent decisions are made. Hence we can conclude that the intelligent sharing of power among microgrids yields more profit than that of the non-sharing case. 

We also observe from the plot that, our model 1 outperforms the model 3. The reason for that is discussed below.  In $ADL-sharing$, there is a flexibility to intelligently schedule the ADL activities according to the non-ADL demand and price. But this is not the case with the $Non-ADL$ model. In this case, the penalty will be immediately levied if the demand (including the ADL demand) is not met. This results in the poor performance of $Non-ADL$. Hence we can conclude that intelligently scheduling the ADL demand results in the better performance.
  


From the above discussion we can conclude that our proposed algorithm along with the flexible ADL demand integration, is the best algorithm that provides more profits to the microgrids. 


\section{Experimental Results}

The following experimental results are desired to be observed:
\begin{itemize}
\item With different ADLs being scheduled along with the non-ADL demand, few of the ADLs are expected to be scheduled at the beginning of the allowed execution time window of the ADL, few other ADLs are expected to be scheduled at the end of the allowed execution time window of the ADL, while some other ADLs get scheduled at the mid of the allowed execution time window. This ensures that the MDP learning agents exploit the fact that the ADL demand is flexible to meet in a given range of time window. On the other-hand, it is not desired that the learning agent schedules all the ADLs either at the beginning or at the end of the allowed time window of execution.
\item With surplus energy available at a microgrid at any moment, it is desired not to sell this surplus to other microgirds if there is more demand than supply in the near feature. For example, if the renewable energy source for a microgrid is solar energy, then if there is surplus energy(i.e., excess energy available after meeting the demand at some moment) at the microgrid during the midday, the microgrid could sell that surplus energy to the other microgrids (because it is expected to generate more supply as the day progresses); on the other-hand, if there is surplus energy at the microgrid during the end of the day, the microgrid might not want to sell that surplus energy to the other microgrids (because there may not be much supply possible for the rest of the day).
%\item How are we ensuring this? If there is 5 units of surplus at time t. If the demand at time (t+1) is 5 units, it's possible to meet that demand by storing 5 units at time t. Other possibility is, sell the 5 units in time t, and buy 5 units in time $t+1$ from some other microgrid. However the first option is most desired. How are we ensuring this in our experiments? One possible way to implement this is by ensuring the buying cost to be more than the selling cost for one unit of energy.   
\end{itemize}

\section{Conclusion}\label{sec:conclusion}
Providing a unified solution framework for modeling both demand-side management problem (scheduling ADL jobs and optimal utilization of storage devices) and supply-side management problem (enabling cooperative energy exchange among the microgrids) is a challenging task, particularly when both demand and supply are considered stochastic. We for the first time in the literature used MDP to integrate these problems into a single unified framework. Also, for the first time in the literature we schedule ADL demand at microgrid level as a load shifting technique. RL algorithms provides optimal solution methodology for solving MDP when the underlying model is not available. We apply Q-learning algorithm to maximize profit earned by microgrids by selling excess energy while maintaining demand and supply gap low. Based on the simulation experiments, we show that our model consistently outperforms other models.

As a future work, we would like to consider the pricing mechanism for microgrids. In the current model, the transaction of power is carried out at the price decided by the main grid. The pricing mechanism allows microgrids to bid for the selling price and buying price. One can use RL agents to bid for adoptive prices in such a way that microgrids maximizes their profits. Another important future work is to use efficient RL algorithms with function approximation technique to scale the proposed algorithms. The challenge here is to select the appropriate features to obtain an optimal policy.

\section*{Acknowledgment}

The authors would like to thank Robert Bosch Centre for Cyber-Physical Systems, IISc, Bangalore, India for supporting part of this work.


 \bibliographystyle{IEEEtran}
 \bibliography{IEEEabrv,reference}

\end{document}