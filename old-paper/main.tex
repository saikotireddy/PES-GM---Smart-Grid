%% bare_conf.tex
%% V1.4a
%% 2014/09/17
%% by Michael Shell
%% See:
%% http://www.michaelshell.org/
%% for current contact information.
%%
%% This is a skeleton file demonstrating the use of IEEEtran.cls
%% (requires IEEEtran.cls version 1.8a or later) with an IEEE
%% conference paper.
%%
%% Support sites:
%% http://www.michaelshell.org/tex/ieeetran/
%% http://www.ctan.org/tex-archive/macros/latex/contrib/IEEEtran/
%% and
%% http://www.ieee.org/

%%*************************************************************************
%% Legal Notice:
%% This code is offered as-is without any warranty either expressed or
%% implied; without even the implied warranty of MERCHANTABILITY or
%% FITNESS FOR A PARTICULAR PURPOSE! 
%% User assumes all risk.
%% In no event shall IEEE or any contributor to this code be liable for
%% any damages or losses, including, but not limited to, incidental,
%% consequential, or any other damages, resulting from the use or misuse
%% of any information contained here.
%%
%% All comments are the opinions of their respective authors and are not
%% necessarily endorsed by the IEEE.
%%
%% This work is distributed under the LaTeX Project Public License (LPPL)
%% ( http://www.latex-project.org/ ) version 1.3, and may be freely used,
%% distributed and modified. A copy of the LPPL, version 1.3, is included
%% in the base LaTeX documentation of all distributions of LaTeX released
%% 2003/12/01 or later.
%% Retain all contribution notices and credits.
%% ** Modified files should be clearly indicated as such, including  **
%% ** renaming them and changing author support contact information. **
%%
%% File list of work: IEEEtran.cls, IEEEtran_HOWTO.pdf, bare_adv.tex,
%%                    bare_conf.tex, bare_jrnl.tex, bare_conf_compsoc.tex,
%%                    bare_jrnl_compsoc.tex, bare_jrnl_transmag.tex
%%*************************************************************************


% *** Authors should verify (and, if needed, correct) their LaTeX system  ***
% *** with the testflow diagnostic prior to trusting their LaTeX platform ***
% *** with production work. IEEE's font choices and paper sizes can       ***
% *** trigger bugs that do not appear when using other class files.       ***                          ***
% The testflow support page is at:
% http://www.michaelshell.org/tex/testflow/



\documentclass[conference]{IEEEtran}
% Some Computer Society conferences also require the compsoc mode option,
% but others use the standard conference format.
%
% If IEEEtran.cls has not been installed into the LaTeX system files,
% manually specify the path to it like:
% \documentclass[conference]{../sty/IEEEtran}

\usepackage{amsmath}
\usepackage{blindtext, graphicx}
\usepackage{algorithm}

\usepackage{bbm}
\usepackage{soul}

\usepackage{algpseudocode}
\usepackage{pifont}
\usepackage{tikz}
\usepackage{amssymb}
%\usepackage{natbib}
\usepackage{upgreek}

\usepackage{mathtools}


% Some very useful LaTeX packages include:
% (uncomment the ones you want to load)


% *** MISC UTILITY PACKAGES ***
%
%\usepackage{ifpdf}
% Heiko Oberdiek's ifpdf.sty is very useful if you need conditional
% compilation based on whether the output is pdf or dvi.
% usage:
% \ifpdf
%   % pdf code
% \else
%   % dvi code
% \fi
% The latest version of ifpdf.sty can be obtained from:
% http://www.ctan.org/tex-archive/macros/latex/contrib/oberdiek/
% Also, note that IEEEtran.cls V1.7 and later provides a builtin
% \ifCLASSINFOpdf conditional that works the same way.
% When switching from latex to pdflatex and vice-versa, the compiler may
% have to be run twice to clear warning/error messages.






% *** CITATION PACKAGES ***
%
%\usepackage{cite}
% cite.sty was written by Donald Arseneau
% V1.6 and later of IEEEtran pre-defines the format of the cite.sty package
% \cite{} output to follow that of IEEE. Loading the cite package will
% result in citation numbers being automatically sorted and properly
% "compressed/ranged". e.g., [1], [9], [2], [7], [5], [6] without using
% cite.sty will become [1], [2], [5]--[7], [9] using cite.sty. cite.sty's
% \cite will automatically add leading space, if needed. Use cite.sty's
% noadjust option (cite.sty V3.8 and later) if you want to turn this off
% such as if a citation ever needs to be enclosed in parenthesis.
% cite.sty is already installed on most LaTeX systems. Be sure and use
% version 5.0 (2009-03-20) and later if using hyperref.sty.
% The latest version can be obtained at:
% http://www.ctan.org/tex-archive/macros/latex/contrib/cite/
% The documentation is contained in the cite.sty file itself.






% *** GRAPHICS RELATED PACKAGES ***
%
\ifCLASSINFOpdf
  % \usepackage[pdftex]{graphicx}
  % declare the path(s) where your graphic files are
  % \graphicspath{{../pdf/}{../jpeg/}}
  % and their extensions so you won't have to specify these with
  % every instance of \includegraphics
  % \DeclareGraphicsExtensions{.pdf,.jpeg,.png}
\else
  % or other class option (dvipsone, dvipdf, if not using dvips). graphicx
  % will default to the driver specified in the system graphics.cfg if no
  % driver is specified.
  % \usepackage[dvips]{graphicx}
  % declare the path(s) where your graphic files are
  % \graphicspath{{../eps/}}
  % and their extensions so you won't have to specify these with
  % every instance of \includegraphics
  % \DeclareGraphicsExtensions{.eps}
\fi
% graphicx was written by David Carlisle and Sebastian Rahtz. It is
% required if you want graphics, photos, etc. graphicx.sty is already
% installed on most LaTeX systems. The latest version and documentation
% can be obtained at: 
% http://www.ctan.org/tex-archive/macros/latex/required/graphics/
% Another good source of documentation is "Using Imported Graphics in
% LaTeX2e" by Keith Reckdahl which can be found at:
% http://www.ctan.org/tex-archive/info/epslatex/
%
% latex, and pdflatex in dvi mode, support graphics in encapsulated
% postscript (.eps) format. pdflatex in pdf mode supports graphics
% in .pdf, .jpeg, .png and .mps (metapost) formats. Users should ensure
% that all non-photo figures use a vector format (.eps, .pdf, .mps) and
% not a bitmapped formats (.jpeg, .png). IEEE frowns on bitmapped formats
% which can result in "jaggedy"/blurry rendering of lines and letters as
% well as large increases in file sizes.
%
% You can find documentation about the pdfTeX application at:
% http://www.tug.org/applications/pdftex





% *** MATH PACKAGES ***
%
%\usepackage[cmex10]{amsmath}
% A popular package from the American Mathematical Society that provides
% many useful and powerful commands for dealing with mathematics. If using
% it, be sure to load this package with the cmex10 option to ensure that
% only type 1 fonts will utilized at all point sizes. Without this option,
% it is possible that some math symbols, particularly those within
% footnotes, will be rendered in bitmap form which will result in a
% document that can not be IEEE Xplore compliant!
%
% Also, note that the amsmath package sets \interdisplaylinepenalty to 10000
% thus preventing page breaks from occurring within multiline equations. Use:
%\interdisplaylinepenalty=2500
% after loading amsmath to restore such page breaks as IEEEtran.cls normally
% does. amsmath.sty is already installed on most LaTeX systems. The latest
% version and documentation can be obtained at:
% http://www.ctan.org/tex-archive/macros/latex/required/amslatex/math/





% *** SPECIALIZED LIST PACKAGES ***
%
%\usepackage{algorithmic}
% algorithmic.sty was written by Peter Williams and Rogerio Brito.
% This package provides an algorithmic environment fo describing algorithms.
% You can use the algorithmic environment in-text or within a figure
% environment to provide for a floating algorithm. Do NOT use the algorithm
% floating environment provided by algorithm.sty (by the same authors) or
% algorithm2e.sty (by Christophe Fiorio) as IEEE does not use dedicated
% algorithm float types and packages that provide these will not provide
% correct IEEE style captions. The latest version and documentation of
% algorithmic.sty can be obtained at:
% http://www.ctan.org/tex-archive/macros/latex/contrib/algorithms/
% There is also a support site at:
% http://algorithms.berlios.de/index.html
% Also of interest may be the (relatively newer and more customizable)
% algorithmicx.sty package by Szasz Janos:
% http://www.ctan.org/tex-archive/macros/latex/contrib/algorithmicx/




% *** ALIGNMENT PACKAGES ***
%
%\usepackage{array}
% Frank Mittelbach's and David Carlisle's array.sty patches and improves
% the standard LaTeX2e array and tabular environments to provide better
% appearance and additional user controls. As the default LaTeX2e table
% generation code is lacking to the point of almost being broken with
% respect to the quality of the end results, all users are strongly
% advised to use an enhanced (at the very least that provided by array.sty)
% set of table tools. array.sty is already installed on most systems. The
% latest version and documentation can be obtained at:
% http://www.ctan.org/tex-archive/macros/latex/required/tools/


% IEEEtran contains the IEEEeqnarray family of commands that can be used to
% generate multiline equations as well as matrices, tables, etc., of high
% quality.




% *** SUBFIGURE PACKAGES ***
%\ifCLASSOPTIONcompsoc
%  \usepackage[caption=false,font=normalsize,labelfont=sf,textfont=sf]{subfig}
%\else
%  \usepackage[caption=false,font=footnotesize]{subfig}
%\fi
% subfig.sty, written by Steven Douglas Cochran, is the modern replacement
% for subfigure.sty, the latter of which is no longer maintained and is
% incompatible with some LaTeX packages including fixltx2e. However,
% subfig.sty requires and automatically loads Axel Sommerfeldt's caption.sty
% which will override IEEEtran.cls' handling of captions and this will result
% in non-IEEE style figure/table captions. To prevent this problem, be sure
% and invoke subfig.sty's "caption=false" package option (available since
% subfig.sty version 1.3, 2005/06/28) as this is will preserve IEEEtran.cls
% handling of captions.
% Note that the Computer Society format requires a larger sans serif font
% than the serif footnote size font used in traditional IEEE formatting
% and thus the need to invoke different subfig.sty package options depending
% on whether compsoc mode has been enabled.
%
% The latest version and documentation of subfig.sty can be obtained at:
% http://www.ctan.org/tex-archive/macros/latex/contrib/subfig/




% *** FLOAT PACKAGES ***
%
%\usepackage{fixltx2e}
% fixltx2e, the successor to the earlier fix2col.sty, was written by
% Frank Mittelbach and David Carlisle. This package corrects a few problems
% in the LaTeX2e kernel, the most notable of which is that in current
% LaTeX2e releases, the ordering of single and double column floats is not
% guaranteed to be preserved. Thus, an unpatched LaTeX2e can allow a
% single column figure to be placed prior to an earlier double column
% figure. The latest version and documentation can be found at:
% http://www.ctan.org/tex-archive/macros/latex/base/


%\usepackage{stfloats}
% stfloats.sty was written by Sigitas Tolusis. This package gives LaTeX2e
% the ability to do double column floats at the bottom of the page as well
% as the top. (e.g., "\begin{figure*}[!b]" is not normally possible in
% LaTeX2e). It also provides a command:
%\fnbelowfloat
% to enable the placement of footnotes below bottom floats (the standard
% LaTeX2e kernel puts them above bottom floats). This is an invasive package
% which rewrites many portions of the LaTeX2e float routines. It may not work
% with other packages that modify the LaTeX2e float routines. The latest
% version and documentation can be obtained at:
% http://www.ctan.org/tex-archive/macros/latex/contrib/sttools/
% Do not use the stfloats baselinefloat ability as IEEE does not allow
% \baselineskip to stretch. Authors submitting work to the IEEE should note
% that IEEE rarely uses double column equations and that authors should try
% to avoid such use. Do not be tempted to use the cuted.sty or midfloat.sty
% packages (also by Sigitas Tolusis) as IEEE does not format its papers in
% such ways.
% Do not attempt to use stfloats with fixltx2e as they are incompatible.
% Instead, use Morten Hogholm'a dblfloatfix which combines the features
% of both fixltx2e and stfloats:
%
% \usepackage{dblfloatfix}
% The latest version can be found at:
% http://www.ctan.org/tex-archive/macros/latex/contrib/dblfloatfix/




% *** PDF, URL AND HYPERLINK PACKAGES ***
%
%\usepackage{url}
% url.sty was written by Donald Arseneau. It provides better support for
% handling and breaking URLs. url.sty is already installed on most LaTeX
% systems. The latest version and documentation can be obtained at:
% http://www.ctan.org/tex-archive/macros/latex/contrib/url/
% Basically, \url{my_url_here}.




% *** Do not adjust lengths that control margins, column widths, etc. ***
% *** Do not use packages that alter fonts (such as pslatex).         ***
% There should be no need to do such things with IEEEtran.cls V1.6 and later.
% (Unless specifically asked to do so by the journal or conference you plan
% to submit to, of course. )


% correct bad hyphenation here
\hyphenation{Smart grids}

\begin{document}

%
% paper title
% Titles are generally capitalized except for words such as a, an, and, as,
% at, but, by, for, in, nor, of, on, or, the, to and up, which are usually
% not capitalized unless they are the first or last word of the title.
% Linebreaks \\ can be used within to get better formatting as desired.
% Do not put math or special symbols in the title.

\title{Smart Grid work for PES GM}


% author names and affiliations
% use a multiple column layout for up to three different
% affiliations

\author{A, B, C}
%\author{Raghuram$^\dagger$, D. Sai Koti Reddy$^\dagger$,  Shalabh Bhatnagar$^\dagger$
%\thanks{
%$^\dagger$ Department of Computer Science and Automation,
%Indian Institute of Science, Bangalore,
%E-Mail: danda.reddy@csa.iisc.ernet.in, shalabh@csa.iisc.ernet.in}
%}

%\author{\IEEEauthorblockN{Raghuram}
%\IEEEauthorblockA{Department of Computer Science and Automation\\
%Indian Institute of Science, Bangalore\\
%Bangalore, India \\
%Email: http://www.michaelshell.org/contact.html}
%\and
%\IEEEauthorblockN{D. Sai Koti Reddy}
%\IEEEauthorblockA{Department of Computer Science and Automation\\
%Indian Institute of Science, Bangalore\\
%Bangalore, India \\
%Email: danda.reddy@csa.iisc.ernet.in}
%\and
%\IEEEauthorblockN{Shalabh Bhatnagar}
%\IEEEauthorblockA{Department of Computer Science and Automation\\
%Indian Institute of Science, Bangalore\\
%Bangalore, India \\
%Email: shalabh@csa.iisc.ernet.in}
%}



% conference papers do not typically use \thanks and this command
% is locked out in conference mode. If really needed, such as for
% the acknowledgment of grants, issue a \IEEEoverridecommandlockouts
% after \documentclass

% for over three affiliations, or if they all won't fit within the width
% of the page, use this alternative format:
% 
%\author{\IEEEauthorblockN{Michael Shell\IEEEauthorrefmark{1},
%Homer Simpson\IEEEauthorrefmark{2},
%James Kirk\IEEEauthorrefmark{3}, 
%Montgomery Scott\IEEEauthorrefmark{3} and
%Eldon Tyrell\IEEEauthorrefmark{4}}
%\IEEEauthorblockA{\IEEEauthorrefmark{1}School of Electrical and Computer Engineering\\
%Georgia Institute of Technology,
%Atlanta, Georgia 30332--0250\\ Email: see http://www.michaelshell.org/contact.html}
%\IEEEauthorblockA{\IEEEauthorrefmark{2}Twentieth Century Fox, Springfield, USA\\
%Email: homer@thesimpsons.com}
%\IEEEauthorblockA{\IEEEauthorrefmark{3}Starfleet Academy, San Francisco, California 96678-2391\\
%Telephone: (800) 555--1212, Fax: (888) 555--1212}
%\IEEEauthorblockA{\IEEEauthorrefmark{4}Tyrell Inc., 123 Replicant Street, Los Angeles, California 90210--4321}}




% use for special paper notices
%\IEEEspecialpapernotice{(Invited Paper)}




% make the title area
\maketitle

% As a general rule, do not put math, special symbols or citations
% in the abstract
\begin{abstract}


One of the key challanges of Smart Grid management is ensuring to meet the demand while having control on the supply cost. \st{This concept of Smart Grid has become popular in the recent times. The main objective in this Smart Grid is to intelligently make use of power.}  It involves \st{taking} performing optimal actions both at the production and consumption sites of \st{electricity} electric grid. In this work, we consider the problem of managing multiple microgrids attached to a central smart grid. Each of these microgrids are equipped with batteries to store renewable power. \st{In this work, we consider multiple microgrids equipped with batteries to store renewable power}. At every instant, each of them receive a demand to meet. Depending on the supply (i.e., currently available battery energy, power drawn either from the central grid or from the peer microgirds), each of the microgrids take a decision on from where to draw the energy to meet the demand. \st{Depending on the current battery and renewable power information, they take a decision on number of power units to be bought or sold.} When a microgrids buys energy either from the central grid or from the peer microgrids, it can use that energy either to meet the current demand or to store in the battery storage for future use. \st{If any of the other neighboring microgrids sell the power, they can consume it. Otherwise, they can get it from the main grid. If power is bought, it is first used to meet its demand and rest of it will be stored in the battery.} Hence, there is a control decision problem at each microgrid on the number of power units to be bought or sold at every time instant. We note that both the forecasted demand and predicted renewable supply impact this decision by each microgrid. \st{We note that the future forecast demand and renewable units also impact this decision.} Further, we consider some amount of the forecasted demand to be adjustable in terms of the time when it can be met by the microgrid. Such an adjustable demand is attributed due to the activities of daily living pertraining to Smart Homes connected to the microgrid networks. Hence, we formulate this problem in the framework of Markov Decision Process and apply Reinforcement Learning algorithms to solve this problem. Through simulations, we show that the policy we obtain performs an significant improvement over traditional techniques.


\end{abstract}

% no keywords

\section{Introduction}


Research on smartgrids can be classified into two areas -  Demand-side management and Supply-side management. Demand side management (DSM) (\cite{logenthiran2011multi, wang2010demand,dsm1,dsm2,dsm3,dsm4}) deals with techniques developed to efficiently use the power by bringing the customers into the play. The main idea is to reduce the consumption of power during peak time and shifting it during the other times. This is done by dynamically changing the price of power and sharing this information with the customers. 

%In \cite{reddy2011learned}, Reinforcement Learning (RL) is used in smart grids for pricing mechanism so as to improve the profits of broker agents who procure energy from power generation sources and sell it to consumers.  
Supply-side management deals with developing techniques to efficiently make use of renewable and non-renewable energy at the supply side. We now discuss the papers that are close to our work. In \cite{PHarsha}, authors consider the problem of optimal energy storage management problem under dynamic cost setup. They consider a renewable generator that is equipped with a limited storage battery and capable of meeting the some local demands. It also has connection from the main grid. The decision to be taken at every instant is the number of power units to be stored in the battery. They allow this to be negative as power can be drawn from the battery. They formulate this problem as a Markov Decision Process under long run average cost. The objective is to minimize the long run average cost of power bought from the main grid.

The assumption made in this paper is that demand at all times can be met. This is facilitated by allowing the main grid as many units of power as asked by the renewable generator. In our work, we extend this MDP model to put additional constraint on the maximum number of power units that a main grid can provide. Another notable contribution is that we extend this setup to the multiple microgrids. Here, we allow the microgrids to share the power among themselves.  

In \cite{reddy2011learned}, they consider the problem of maximizing the profits among the broker agents. These agents buy the power from the producers and sell it to the customers. They apply Reinforcement Learning (RL) algorithms to solve this problem to show that learning policies perform better than the traditional non-learning strategies.

In \cite{goodmdp}, authors propose an MDP for solving the problem of minimizing the demand and supply deficit  
and apply dynamic optimization methods. But when the model information (the renewable energy generation in this case) is not known, we cannot apply these techniques.

\section*{Organization of the Paper}	
The rest of the paper is organized as follows. The next section describes the important problems associated with the microgrids and solution techniques to solve them. Section III presents the results of experiments of our algorithms. Section IV provides the concluding remarks and Section V discusses the future research directions.


% For peer review papers, you can put extra information on the cover
% page as needed:
% \ifCLASSOPTIONpeerreview
% \begin{center} \bfseries EDICS Category: 3-BBND \end{center}
% \fi
%
% For peerreview papers, this IEEEtran command inserts a page break and
% creates the second title. It will be ignored for other modes.
\IEEEpeerreviewmaketitle



 


% \subsection{Subsection Heading Here}
% Subsection text here.


% \subsubsection{Subsubsection Heading Here}
% Subsubsection text here.

\section{Problem and Model}

Consider multiple microgrids deployed at the sites where renewable energy is available. They are provided with the limited storage batteries that can store the power. Note that they do not have power generation capabilities. They have electrical connections from neighboring microgrids and also from the main grid. But the sharing of power among the microgrids is the most preferred mode as it results in the minimum dissipation loss of power. Only if the sharing among the microgrids is not possible, the main grid connection comes into play. 

At the beginning of every time instant (For ex. every hour of a day), each microgrid obtains the demand that they need to meet. They also obtain the renewable power generated in that instant. Based on these two information, along with the battery information, the microgrids need to take a decision on number of power units to be bought/sold. If the power is bought, it is first used to meet the demand and the remaining will be stored in the battery. One may think of a simple greedy policy to solve this problem. That is, the microgrids sells the excess units and buys when there is deficit of demand. But this behaviour may not result in the optimal behavior in a long-run. For example, consider a situation where a microgrid have an surplus of power. But the forecasted demand in the future is very high and the forecasted renewable energy is very low. In this case, having power in its battery will be very helpful. Otherwise, it has to buy the power from the main grid which can be very high. Hence the microgrid needs to balance the current and future rewards. Mathematically, we can formulate this problem in the framework of Markov Decision Process (MDP).

*** Write about MDP, Infinite horizon and Average cost MDP *******

Now we formally present the MDP model for our problem. Let us denote $x_{t}^{i} = (d_{t}^{i},r_{t}^{i},b_{t}^{i})$  as the demand, renewable generation and battery information of the microgrid $i$ at time instant $t$. Let the maximum size of the battery be $B$. We assume that the Demand is a Markov Process. That is, the demand in the next time period depends only on the current demand. Renewable generation is assumed to be a stochastic process. As described earlier, the microgrids can share the power among themselves. This is done at the price decided by the main grid. We denote the price of a unit of power at time instant $t$ as $p_{t}$. The price is also modeled as a Markov Process. Thus the state space is denoted by $s_{t}^{i} = <t,x_{t}^{i},p_{t}>$. Note that we include the current time also in the state space. This is because for the same $x_{t}^{i}$ and $p_{t}$, the decision taken can be different at different times. For example, a microgrid operating on solar renewable will be able to sell the power during the morning time as they may still receive the solar power during the afternoon. But they may not want to sell the power during the evening as they will be no solar power in the night. Instead they store the excess power in the battery.

The decision taken by the microgrid $i$ at the time $t$ is denoted as $u_{t}^{i}$. This can be positive or negative. Positive action denotes the number of units that the microgrid is willing to sell. Negative units represents the number of units that the microgrid is willing to buy. If a microgrid needs power, it first accepts it from the neighboring microgrids. Then the remaining units (if any) will be bought from the main grid. If any microgrid has a surplus power that is not consumed by any other, it sells it to the main grid. To maintain the stability at the main grid, We impose a constraint on the number of units of power $M$ that a main grid can give to each microgrid. 

We observe that the action to be taken depends only on the net demand. That is we can further simplify the state space to be $s_{t}^{i} = (t,nd_{t}^{i},p_{t})$ where $nd_{t}^{i} = r_{t}^{i} + b_{t}^{i} - d_{t}^{i}$. If this is negative, it means there is a deficit in demand and positive implies there is excess of power.

Then the action is bounded as follows. For ease of understanding, we drop the subscripts.  

\begin{align}
-min(M, B - nd) \leq u \leq max(0, nd)
\end{align}

The intuition behind the bounds is as follows. A microgrid can sell atmost the excess power. That is, the power remaining after meeting the demand. While buying, it can buy to meet the demand and also to fill its battery.

The battery information is updated as follows:

\begin{align}
b_{t+1}^{i} = max(0,nd_{t}^{i} - u_{t}^{i})
\end{align}

We formulate the single stage cost function as follows :
\begin{align}
g^{i}(s,u) = p_{t}*u_{t}^{i} + c*(min(0,nd_{t}^{i} - u_{t}^{i}).
\end{align}

The first term represents the cost of buying or selling the power and the second term represents the demand and supply deficit. For every unit of demand that is not met, the microgrid incurs a cost of $c$. 

Finally, the objective of the microgrid $i$ is to maximize the following \cite{avgcost}:

\begin{align}
limsup_{n \rightarrow \infty} 1/n \sum_{k = 0}^{n} E(g^{i}(s_{k},u_{k})),
\end{align}

where $E(.)$ is the expectation. 

We also consider the long run discounted cost formulation. The objective here is to maximize the following:

\begin{align}
limsup_{n \rightarrow \infty} \sum_{k = 0}^{n} \gamma^{k} * E(g^{i}(s_{k},u_{k})),
\end{align}

where $\gamma$ is the discount factor. 


\section{Experimental Results}

\section{Conclusion}
We considered the problem of electrifying a village by setting up microgrids close to the village. These microgrids have access to the renewable energy storage batteries and also electrical connections from the main grid. We identified two problems associated with the microgrid. First problem is to minimize the expected long-run discounted demand-supply deficit. We model this problem in the framework of MDP \cite{goodmdp}. This formulation doesn't take into consideration the cost of power production at the main grid site. Finally, we formulated MDP taking the cost of power production into consideration. We applied Multi-Agent Q-Learning algorithm to solve these problems. Simulations show that, when maximum power allocation at the main site is not very less, storing the power in the batteries and using them intelligently is the better solution compared to not using the storage batteries. 

\section{Future Work}
As future work, we would like to consider the possibility of power sharing between the microgrids. In this case, along with decision on amount of power to be used from stored batteries, microgrids also have to make decision on the amount of power that can be shared with others. We would also like to consider the  heterogeneous power price system. In this scenario, the price of power production at the main grid will vary from time to time. 

\section*{Acknowledgment}

The authors would like to thank Robert Bosch Centre for Cyber-Physical Systems, IISc, Bangalore, India for supporting part of this work.


% conference papers do not normally have an appendix


% use section* for acknowledgment
%\section*{Acknowledgment}

%The authors would like to thank Robert Bosch Centre for Cyber-Physical Systems, IISc, Bangalore, India for supporting this work.


 \bibliographystyle{IEEEtran}
 \bibliography{IEEEabrv,reference}




% that's all folks
\end{document}