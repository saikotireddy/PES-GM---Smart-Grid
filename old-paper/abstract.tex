\begin{abstract}


One of the key challanges of Smart Grid management is ensuring to meet the demand while having control on the supply cost. \st{This concept of Smart Grid has become popular in the recent times. The main objective in this Smart Grid is to intelligently make use of power.}  It involves \st{taking} performing optimal actions both at the production and consumption sites of \st{electricity} electric grid. In this work, we consider the problem of managing multiple microgrids attached to a central smart grid. Each of these microgrids are equipped with batteries to store renewable power. \st{In this work, we consider multiple microgrids equipped with batteries to store renewable power}. At every instant, each of them receive a demand to meet. Depending on the supply (i.e., currently available battery energy, power drawn either from the central grid or from the peer microgirds), each of the microgrids take a decision on from where to draw the energy to meet the demand. \st{Depending on the current battery and renewable power information, they take a decision on number of power units to be bought or sold.} When a microgrids buys energy either from the central grid or from the peer microgrids, it can use that energy either to meet the current demand or to store in the battery storage for future use. \st{If any of the other neighboring microgrids sell the power, they can consume it. Otherwise, they can get it from the main grid. If power is bought, it is first used to meet its demand and rest of it will be stored in the battery.} Hence, there is a control decision problem at each microgrid on the number of power units to be bought or sold at every time instant. We note that both the forecasted demand and predicted renewable supply impact this decision by each microgrid. \st{We note that the future forecast demand and renewable units also impact this decision.} Further, we consider some amount of the forecasted demand to be adjustable in terms of the time when it can be met by the microgrid. Such an adjustable demand is attributed due to the activities of daily living pertraining to Smart Homes connected to the microgrid networks. Hence, we formulate this problem in the framework of Markov Decision Process and apply Reinforcement Learning algorithms to solve this problem. Through simulations, we show that the policy we obtain performs an significant improvement over traditional techniques.


\end{abstract}